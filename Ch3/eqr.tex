In this (sub)section we will explore a special kind of relation called an \textbf{equivalence relation}. Suppose $A$ is a set. Roughly speaking, there are two useful ways of viewing an equivalence relation $A$.
\begin{itemize}
	\item We can view the relation as a classifier. That is, the relation divides the elements of $A$ into groups and classifies them as being the same if they are in the same group.
	\item Alternatively (and equivalently), the relation compares two elements $a$ and $b$ and determines whether the two elements are ``the same'' or not the same.
\end{itemize}

In this section we will see how these two views describe the same thing. In short this relationship can be more formally described as how we can use an equivalence relation to partition $A$ into several \textit{equivalence classes}. We will fully define these terms more formally later. First we begin with the definition of an equivalence relation. 

\begin{definition}
Let $A$ be a set. An equivalence relation $\sim \subset A \times A$ is a relation with the following properties:
\begin{enumerate}
	\item For all $a \in A$, $a \sim a$.
	\item If $a \sim b$ for $a, b \in A$, then $b \sim a$.
	\item If $a \sim b$ and $b \sim c$, then $a \sim c$. 
\end{enumerate}


\end{definition}

Let us unpack the criterion for an equivalence relation $\sim$ on $A$. The first condition simply states that every element in $a$ is related to itself. The second condition models the condition that if $a$ is related to $b$, then $b$ should be related to $a$ as well. The third condition asserts the \textit{transitivity} of the relation $\sim$. That is, if $a$ is related to $b$, and $b$ is related to $c$, then $a$ should be related to $c$; that is, we can chain relations to get other relations.

\begin{example}
The notion of object equality is an equivalence relation. More formally, the $=$ symbol can be seen as a relation on any set $A$. We check that this satisfies all the properties that an equivalence relation should satisfy:
\begin{enumerate}
	\item $a = a$ for all $a \in A$.
	\item If $a = b$, then $b = a$ as well.
	\item If $a = b$ and $b = c$, then $a = c$ as well.
\end{enumerate}
\end{example}
The verifications we made above are true because they are in fact properties of \textit{equality} itself. We remark that in fact the definition of an equivalence relation was formed to encode more general notions of equality. We will describe more examples of equivalence relations later in the section and explore for each relation in what way the elements are the same. 

\begin{example}
For the set $\mathbb{Z}$ define a relation $\sim$ as follows. We have $a \sim b$ if $a - b$ is even. We can verify that $a \sim b$ only if $a$ and $b$ are both odd or both even. So in this case, $\sim$ defines whether or not items are the same based on whether or not they have the same remainder when integer dividing by $2$.
\end{example}