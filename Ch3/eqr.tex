In this (sub)section we will explore a special kind of relation called an \textbf{equivalence relation}. Suppose $A$ is a set. Roughly speaking, there are two useful ways of viewing an equivalence relation $A$.
\begin{itemize}
	\item We can view the relation as a classifier. That is, the relation divides the elements of $A$ into groups and classifies them as being the same if they are in the same group.
	\item Alternatively (and equivalently), the relation compares two elements $a$ and $b$ and determines whether the two elements are ``the same'' or not the same.
\end{itemize}

In this section we will see how these two views describe the same thing. In short this relationship can be more formally described as how we can use an equivalence relation to partition $A$ into several \textit{equivalence classes}. We will fully define these terms more formally later. First we begin with the definition of an equivalence relation. 

\begin{definition}
Let $A$ be a set. An equivalence relation $\sim \subset A \times A$ is a relation with the following properties:
\begin{enumerate}
	\item For all $a \in A$, $a \sim a$.
	\item If $a \sim b$ for $a, b \in A$, then $b \sim a$.
	\item If $a \sim b$ and $b \sim c$, then $a \sim c$. 
\end{enumerate}

Moreover, these conditions have specific names.
\begin{enumerate}
	\item Any relation satisfying the first condition is called \textit{reflexive}.
	\item Any relation satisfying the second condition is called \textit{symmetric}.
	\item Any relation satisfying the third condition is called \textit{transitive}.
\end{enumerate}
\end{definition}

Let us unpack the criterion for an equivalence relation $\sim$ on $A$. The first condition simply states that every element in $a$ is related to itself. The second condition models the condition that if $a$ is related to $b$, then $b$ should be related to $a$ as well. The third condition asserts the \textit{transitivity} of the relation $\sim$. That is, if $a$ is related to $b$, and $b$ is related to $c$, then $a$ should be related to $c$; that is, we can chain relations to get other relations.

\begin{example}
The notion of object equality is an equivalence relation. More formally, the $=$ symbol can be seen as a relation on any set $A$. We check that this satisfies all the properties that an equivalence relation should satisfy:
\begin{enumerate}
	\item $a = a$ for all $a \in A$.
	\item If $a = b$, then $b = a$ as well.
	\item If $a = b$ and $b = c$, then $a = c$ as well.
\end{enumerate}
\end{example}
The verifications we made above are true because they are in fact properties of \textit{equality} itself. In fact, the definition of an equivalence relation was formed to encode more general notions of equality. We will describe more examples of equivalence relations later in the section and explore for each relation in what way the elements are the same. 

\begin{example}\label{apt1_eqr}
Suppose we have an apartment building $A$. Let $S$ be the set of all residents in $A$. We will define an equivalence relation $\sim$ on $S$ as follows. For $s, t \in S$ (ie, any two residents of the building), we say that $s \sim t$ if $s$ and $t$ live at the same apartment number. We can verify that this is an equivalence relation by verifying the three properties an equivalence relation needs to satisfy:
\begin{enumerate}
	\item For any $s \in S$, $s \sim s$ by default because they are the same person and hence they live at the same apartment number.
	\item If $s$ and $t$ live in the same apartment number this is the same as saying that $t$ and $s$ live at the same apartment number. So if $s \sim t$, then $t \sim s$.
	\item If $s \sim t$ and $t \sim r$, then $s$, $t$, and $r$ all live at the same apartment number. It follows that $s \sim r$ as desired.
\end{enumerate}
\end{example}

\begin{example}\label{even_odd_eqr}
For the set $\mathbb{Z}$ define a relation $\sim$ as follows. We have $a \sim b$ if $a - b$ is even. We can verify that this relation is an equivalence relation below:
\begin{enumerate}
	\item $a-a = 0$, which is even (it is $2 \times 0$), so $a \sim a$.
	\item If $a \sim b$, then $a - b$ is even. It follows that $b - a = -(a-b)$ is even as well. For if $a - b = 2m$ for some integer $m$ then $b - a = -(a-b) = -2m = 2(-m)$ can be seen to be an even integer. So $b \sim a$.
	\item If $a \sim b$ and $b \sim c$, then $a - b$ and $b - c$ are both even. So $a - b = 2m$ and $b - c = 2n$ for some integers $m$ and $n$. It can be seen that their sum must be even as well. But we have
	\[a - c = (a - b) + (b - c) = 2m + 2n = 2(m + n)\]
	so it follows that $a - c$ is even as well. Hence $a \sim c$ and the relation is transitive.
\end{enumerate}

So it follows that $\sim$ is an equivalence relation. We can verify that $a \sim b$ only if $a$ and $b$ are both odd or both even. So in this case, $\sim$ defines whether or not items are the same based on whether or not they have the same remainder when integer dividing by $2$.

We will also show that a similar looking relation is not an equivalence relation. Suppose instead we define $a \sim b$ if $a \sim b$ is odd. One can check that this relation is symmetric, but it is neither reflexive nor transitive.
\end{example}

\begin{example}\label{modm_eqr}
We can generalize the previous example as follows. For the set $\mathbb{Z}$ we can define a relation $\sim$ with the property that $a \sim b$ if $b - a$ is divisible by a positive integer $m$ (where $m$ is given in advance). One can check as above that this forms an equivalence relation.
\end{example}

At the beginning of this (sub)section we described two different ways of looking at an equivalence relation. We have made the second way (an equivalence relation as determining whether two elements are the ``same'' in some sense) above. We now make the first notion (an equivalence relation as a classifier) precise.

\begin{definition}
Suppose $S$ is a set. Then a partition $P$ of $S$ is a collection of subsets $P_n$ (possibly infinite) where any two $P_i$ and $P_j$ are mutually disjoint and $\bigcup_{j = 1}^nP_j = S$. 
\end{definition}

One can use the word ``partition'' either as a noun indicating a set partition, as defined above, or as a verb describing the action of dividing a set into disjoint subsets. In the example below ``partition'' is used in its verb form.

\begin{example}
We can partition the set $S = \{1, 2, 3, 4, 5, 6\}$ into the even elements $E = \{2, 4, 6\}$ and odd elements $O = \{1, 3, 5\}$. One can easily see that $E$ and $O$ are disjoint and $E \cup O = S$.
\end{example}

Now we will describe how given an equivalence relation we can partition a set in a natural way. The idea is to include elements in the same subset if they are related to each other. A typical example is the previous example. If $\sim$ represents the equivalence relation $a \sim b$ if $a - b$ is even, then all the elements in $E$ are equivalent and all the elements in $O$ are equivalent. To generalize this notion to any equivalence relation, we will introduce the notion of an \textit{equivalence class}.

\begin{definition}
Suppose that $S$ is a set and $\sim$ is an equivalence relation on $S$. Let $a \in S$ be any element. We define 
\[C_a = \{s \in S \mid s \sim a\}.\]
In other words, $C_a$ is the subset of $S$ whose elements are all equivalent to $a$.
\end{definition}

\begin{example}\ref{even_odd2_eqr}
In example \ref{even_odd_eqr} we defined an equivalence relation on $\mathbb{Z}$ with $a \sim b$ if $a - b$ was an even number. For this example we consider what $C_0$ is. If $a \sim 0$, then $a - 0 = a$ is an even number, and conversely. So $C_0 = \{\dots,-4, -2, 0, 2, 4, \dots,\}$. Similarly, we see that $C_1 = \{\dots,-3, -1, 1, 3, \dots\}$. We might observe the following:
\begin{itemize}
	\item $C_0$ and $C_1$ are disjoint. Their union is all of $\mathbb{Z}$, so $C_0$ and $C_1$ form a partition of the integers.
	\item We might notice that $C_4$, $C_2$, and $C_0$ are all the same sets. Indeed, we can convince ourselves that in this example, no matter what element $i$ we take from $C_0$, it will be the case that $C_i = C_0$. So an equivalence class seems to not depend on the \textit{choice of representative}.
\end{itemize}
\end{example}

Let's further explain what we mean by a \textit{choice of representative}. In defining an equivalence class $C_a$, we make a choice of element $a$. But as the above example demonstrates, it seems like we can choose any element from an equivalence class and the resulting equivalence class will turn out to be same. In fact, this happens for any general equivalence relation and we will explain why below. The reasoning follows from a particular claim:

\begin{proposition}
Let $S$ be a set, and let $\sim$ be an equivalence relation on $S$. For any $a, b \in S$, two possibilities happen.
\begin{enumerate}
	\item $C_a \cap C_b = \varnothing$, or the two equivalence classes are disjoint.
	\item $C_a = C_b$, ie, the two equivalence classes are the same. 
\end{enumerate}
\end{proposition}

To show that this claim is true, we examine the two subsets $C_a$ and $C_b$. Maybe it is the case that $C_a \cap C_b = \varnothing$. Then this possibility has happened and there is nothing left to show. The other possibility is that $C_a \cap C_b \neq \varnothing$. Then to show that the claim is true we need to show that in fact $C_a = C_b$.

First we show that $a \sim b$. Since $C_a \cap C_b \neq \varnothing$ there is an element $c \in C_a \cap C_b$. By definition of set intersection, $c \in C_a$, and also $c \in C_b$. By definition of $C_a$ and $C_b$, we have that 
\[c \sim a, c \sim b.\]
By symmetry of the equivalence relation, $a \sim c$, and hence by transitivity we have $a \sim b$.

Now that we know that $a \sim b$ it is pretty easy to show that $C_a \subset C_b$ and vice versa. To show that $C_a \subset C_b$, suppose $d$ is any element of $C_a$. By definition, $d \sim a$. But since $a \sim b$, by transitivity, $d \sim b$. So $d \in C_b$. It follows that $d \in C_b$. So $C_a \subset C_b$ as desired. To show $C_b \subset C_a$ is the same reasoning except we use the fact that $b \sim a$ (which is true by the symmetry property, since $a \sim b$).

%%%still need to explain how this gives you a partition of S
We are almost done. Now we just need to answer the following question: how does this claim help us generate a partition of the set $S$? To answer this partially we might describe how to partition $S$ using an iterative process:
\begin{enumerate}
	\item Choose any element $a \in S$. We take $C_a$.
	\item Choose any element $b \in S$, $b \not\in C_a$. This ensures that $C_a \cap C_b = \varnothing$ (why?). 
	\item Choose any element $c \in S$, $c \not\in C_a \cup C_b$. This ensures that $C_c$ is disjoint from the first two sets. 
	\item Keep going like this until $S$ is exhausted of elements.
\end{enumerate}
In more informal terms, we think about $S$ as a big box of elements, and we keep taking out elements of $S$ in bags (which are the equivalence classes), until $S$ is empty. The reason why this is not quite complete is that of course $S$ may be infinite, and the equivalence classes finite, so of course one might never reach a complete partition of $S$ by doing this. But the idea is still the same.

To partition any general set $S$ with an equivalence relation $\sim$ we first observe that the following identity holds:
\[S = \bigcup_{a \in S}C_a.\]
To see why this is true, we first observe that since all the $C_a \subset S$, then the backwards inclusion is true, ie $S \supset \bigcup_{a \in S}C_a$. For the forwards inclusion we observe that for any $c \in S$, $c \in C_c$ because $c \sim c$ by reflexivity. It follows that the forward inclusion holds, and hence the set equality holds.

Now we are tempted to say that the collection $\{C_a\}_{a \in S}$ is the required partition, because by the claim we showed above, the $C_a$ are either the same or disjoint. However, since some of the $C_a$ may be the same this is not generally a partition yet. What we do now is consider all collections $C_a, C_b, \dots$ which are the same, and discard all but one of them. After this process is completed we have our required partition.

\begin{example}\label{apt2_eqr}
In Example \ref{apt1_eqr}, we showed that the equivalence relation $p \sim q$ on two apartment building residents if $p$ and $q$ live at the same apartment number was an equivalence relation. The equivalence class $C_p$ consists of the set of all people who are roommates with $p$. In general, by our reasoning above, the collection of distinct and disjoint equivalence classes partitions the set of apartment residents $A$. Ignoring the representative of any equivalence class, we see that any subset used in our partition is just the collection of all the people living in a particular apartment number. 

If we are to view this equivalence relation as a classifier, then roughly speaking this equivalence relation classifies people based on what apartment number they reside at. 
\end{example}

\begin{example}\label{mod2_classes_eqr}
In Example \ref{even_odd2_eqr} we divided the integers $\mathbb{Z}$ based on whether they were odd or even. So in terms of a classifier this equivalence relation classifies integers based on whether or not they are even or odd.
\end{example}

\subsubsection{Exercises}

\begin{enumerate}
	\item Let $X$ be any non-empty set, and consider the relation $R = X \times X \subseteq X \times X$. Verify that this relation is an equivalence relation.
	\item Give an example of a relation $R$ on a set $A$ that is
\begin{itemize}
    \item reflexive and symmetric but not transitive,
    \item symmetric and transitive but not reflexive,
    \item reflexive and transitive but not symmetric.
\end{itemize}
	\item Suppose Lily presents to you the following argument that a relation having the transitive and symmetric property is in fact reflexive: they explain that since $a \sim b$, then by symmetry, $b \sim a$. But then by transitivity, $a \sim a$. So the relation must be reflexive. Is Lily right? If not, what is wrong with their argument? 
	\item (The classification of symmetric, transitive relations.) Let $A$ be a non-empty set, and suppose that $R$ is a non-empty relation which is symmetric and transitive. Show there is a non-empty set $B \subseteq A$ for which $R$ is an equivalence relation restricted to $B \times B$. Explicitly describe the set $B$. (It might help to do the second part of problem 2 first.)
	\item (Equivalence relations as surjective functions) Let $X$ be a set, and let $\sim~\subset X \times X$ be an equivalence relation. Given any $a \in X$, define
\[X_a = \{b \in X \mid a\sim  b\}\]
We call this the equivalence class of $a$.

\begin{enumerate}
    \item Given any $a, b \in X$, show that either $X_a \cap X_b = \varnothing$ or $X_a = X_b$. So distinct equivalence classes are disjoint.
    \item Let $C = \{X_a \mid a \in A\}$. That is, $C$ is the set of all the equivalence classes (which are sets, specifically subsets of $X$). Consider the function $f:X \to C$ given by $f(a) = X_a$. First, justify that $f$ is actually a function (that is, it satisfies all the properties that functions satisfy). Then prove that $f$ is a surjective function (this is pretty easy).
    \item Now consider any surjective function
    \[f:X \to C,\] where $C$ is any set. Consider the sets 
    \[M_z = \{x \in X \mid f(x) = z\}.\]
    Show that the relation $\sim~\subset X \times X$ defined by $x \sim y$ if and only if $f(x) = f(y)$ is an equivalence relation. Show that the equivalence classes of $\sim$ are precisely the sets $M_z$ for $z \in C$.
    \item Let $D$ be the set of equivalence classes of $\sim$ in the previous part. Now define $\tilde{f}: D \to C$ by $\tilde{f}(X_a) = f(a)$.
    Show that $\tilde{f}$ with this rule defines a valid function and that $\tilde{f}$ is in fact bijective.
\end{enumerate}
Remark: for the more mathematically inclined the reason why this construction works boils down to fact that the inverse image of a function is well behaved under set union and intersection; see if you can figure out why this fact basically makes the problem work.
\end{enumerate}