In this chapter we discuss the notion of a \textbf{set} and various terminology associated with it. In many theoretical areas such as mathematics, computer science, and physics, the notion of a set comes up very naturally (in general, it comes up when one is considering any collection or class of objects). In many fields, especially in technical literature, such knowledge about sets and their associated terminology are basic knowledge that everyone should know.

Here are the topics that we will cover:
\begin{itemize}
	\item First we will have a reasonable discussion about what a set is, and then describe some common set operations. We will discuss some of the relationships between some of these operations. 
	\item After this we will discuss the \textit{function}, which is arguably a more important concept than the sets themselves. We will describe some common properties that functions have and discuss some common operations, such as function composition. 
	\item We will explore relations, which are a way of comparing elements in any given set, and classify certain nice relations called equivalence relations.
	\item Finally, we will explore the notion of ``size'' of a set, where we demonstrate that there are different types of ``infinity''.
\end{itemize}