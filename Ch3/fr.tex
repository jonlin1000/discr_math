\section{Functions}

\subsection{Definition and Terminology}
In this section, we will answer two questions. Namely, what is a function, and what do functions do?

From prior experience in precalculus and calculus courses most people have an intuitive picture of what functions are. For most purposes, they are ``rules'' that sends values to other values. In other words:

\begin{tcolorbox}
A function is a rule that assigns to certain elements in a set, certain elements to another set.
\end{tcolorbox}

The word rule is not very clear or rigorous. For example, we might consider the real valued functions $f(x) = x^3$ and $g(x) = x^3 + 3x - 3(x + 3) + 9$. These have the same value everywhere, but the rule for evaluating them is slightly different.

It turns out that when defining functions rigorously, the ``rule'' is a means to an end. All we care about when talking about any specific function is the output given the input. 

\begin{definition}
A function $f:A \to B$ (read as ``$f$ from $A$ to $B$'') is defined as a set of ordered pairs $(a, b)$ where $a \in A$ and $b \in B$ (more specifically, a subset of the set of ordered pairs $A \times B$). $f$ has the following properties:
\begin{enumerate}
    \item For all $a \in A$, there exists some $b \in B$ such that $(a, b) \in f$.
    \item For all $a \in A$, $b, c \in B$, if $(a,b) \in f$ and $(a, c) \in f$, then $b = c$.
\end{enumerate}
\end{definition}

In other words, every value $a \in A$ will have some value $b$ such that $(a, b)$ is in $f$. Moreover, this value is \textbf{unique}. We define $f(a)$ as the unique value in $B$ such that $(a, f(a)) \in f$.

Here is some more terminology.
\begin{definition}
Let $f: A \to B$ be a function. We say $A$ is the \textbf{domain} of the function $f$ and $B$ is the \textbf{codomain} of $f$. We say that $a$ is mapped to $b$ or $f$ maps $a$ to $b$ if $(a, b) \in f$. You can also write $a \mapsto b$ as long as it is clear what $f$ is.
\end{definition}

Using these definitions, we can say the following: Given any function $f:A \to B$, every value in the domain will be mapped to exactly one value in the codomain.

\subsection{Injectivity, Surjectivity, Bijectivity}

Here are the relevant definitions:
\begin{itemize}
    \item A function $f:A \to B$ is injective (or one to one) if $f(a) = f(b) \implies a = b$. So two different values in $A$ do not map to the same value in $B$.
    \item A function $f:A \to B$ is surjective (or onto) if for all $b \in B$ we can find a value in $A$ such that $f(a) = b$.
    \item A function $f$ is bijective if it is both injective and surjective.
\end{itemize}

Here is an interesting observation: if $f: A \to B$ is a bijection, then there is an inverse function $g: B \to A$. For each $b \in B$, we can define $g(b)$ as the unique value $a \in A$ such that $f(a) = b$. We know such a value exists because $f$ is surjective, and we know that this value is unique because $f$ is injective.

\section{Relations}

Let $A$ be a set. Traditionally, we have lots of different notation for when we might want to \textit{compare} two objects in $A$. For example, the $=$ notation ``compares'' for equality, and the $\leq$ symbol compares two objects for magnitude. Relations generalize these comparison operators.

\begin{definition}
Let $A$ and $B$ be sets. A relation $R$ on $A$ and $B$ is a subset of $A \times B$. That is, $R$ is any subset of ordered pairs $(a, b)$ where $a \in A$ and $b \in B$.
\end{definition}

For example, a function $f$ is a relation.

Most of the time we will consider relations in the special case where $A = B$. This is, as we might observe, the use case for the equality and inequality operators mentioned at the beginning of this section.

\begin{definition}
Let $A$ be a set and $R$ be a relation on $A \times A$. A relation is
\begin{itemize}
    \item \textit{reflexive} if $(a, a) \in R$ for every $a \in A$,
    \item \textit{symmetric}, if for all $a, b \in A$, $(a, b) \in R \implies (b, a) \in R$.
    \item \textit{transitive} if for all $a, b, c \in A$, $(a, b) \in R, (b, c) \in R \implies (a, c) \in R$.
\end{itemize}

We write $aRb$ if $(a, b) \in R$.

A relation is called an \textbf{equivalence relation} if it is reflexive, symmetric, and transitive.
\end{definition}

\section{Exercises}

 \begin{enumerate}
   \item Let $X$ be any non-empty set, and consider the relation $R = X \times X \subseteq X \times X$. Verify that this relation is an equivalence relation.
   \item Give an example of a relation $R$ on a set $A$ that is
\begin{itemize}
    \item reflexive and symmetric but not transitive,
    \item symmetric and transitive but not reflexive,
    \item reflexive and transitive but not symmetric.
\end{itemize}
   \item Suppose $f:A \to B$ and $g:B \to A$ are functions such that 
\[g(f(a)) = a\] for all $a \in A$. Show that $f$ is injective and that $g$ is surjective.
   \item Let $f$ and $g$ be functions as in the last problem. Suppose also that $f(g(b)) = b$ for all $b \in B$. Show that $g$ is the only function with these properties, that is if $h$ has these properties then $h = g$. (Notice that this equality is technically realized as an equality of \textbf{sets}.)
   \item (The classification of symmetric, transitive relations.) Let $A$ be a non-empty set, and suppose that $R$ is a non-empty relation which is symmetric and transitive. Show there is a non-empty set $B \subseteq A$ for which $R$ is an equivalence relation restricted to $B \times B$. Explicitly describe the set $B$. (It might help to do the second part of problem 2 first.)
   \item (Equivalence relations as surjective functions) Let $X$ be a set, and let $\sim~\subset X \times X$ be an equivalence relation. Given any $a \in X$, define
\[X_a = \{b \in X \mid a\sim  b\}\]
We call this the equivalence class of $a$.

\begin{enumerate}
    \item Given any $a, b \in X$, show that either $X_a \cap X_b = \varnothing$ or $X_a = X_b$. So distinct equivalence classes are disjoint.
    \item Let $C = \{X_a \mid a \in A\}$. That is, $C$ is the set of all the equivalence classes (which are sets, specifically subsets of $X$). Consider the function $f:X \to C$ given by $f(a) = X_a$. First, justify that $f$ is actually a function (that is, it satisfies all the properties that functions satisfy). Then prove that $f$ is a surjective function (this is pretty easy).
    \item Now consider any surjective function
    \[f:X \to C,\] where $C$ is any set. Consider the sets 
    \[M_z = \{x \in X \mid f(x) = z\}.\]
    Show that the relation $\sim~\subset X \times X$ defined by $x \sim y$ if and only if $f(x) = f(y)$ is an equivalence relation. Show that the equivalence classes of $\sim$ are precisely the sets $M_z$ for $z \in C$.
    \item Let $D$ be the set of equivalence classes of $\sim$ in the previous part. Now define $\tilde{f}: D \to C$ by $\tilde{f}(X_a) = f(a)$.
    Show that $\tilde{f}$ with this rule defines a valid function and that $\tilde{f}$ is in fact bijective.
\end{enumerate}
Remark: for the more mathematically inclined the reason why this construction works boils down to fact that the inverse image of a function is well behaved under set union and intersection; see if you can figure out why this fact basically makes the problem work.
   % add more problem files here
 \end{enumerate}

