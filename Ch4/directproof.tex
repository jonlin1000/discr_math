Here are two examples of direct proofs.
\begin{proposition}
$5$ is an odd number.
\end{proposition}
\begin{proof}
Since we can write $5 = 2(2) + 1$, it follows that $5$ is odd.
\end{proof}

\begin{proposition}
The sum of two odd numbers is even.
\end{proposition}
\begin{proof}
Suppose $m$ and $n$ are odd numbers. Then we may write
\begin{gather*}
m = 2k + 1 \\
n = 2j + 1
\end{gather*}
where $k$ and $j$ are integers. Then
\[m + n = 2k + 2j + 2 = 2(k + j + 1)\]
is even, since $k + j + 1$ is an integer.
\end{proof}

The claims were kept simple to better indicate their structure. Notice that the first claim has the ``object has property'' form. The second claim has the ``hypothesis implies conclusion'' form, once we rewrite the claim in the following form:
\begin{proposition}
If $m$ and $n$ are odd numbers, then their sum $m + n$ is an even number.
\end{proposition}
Notice how the proof of this claim takes two unspecified odd numbers and verifies that their sum is even. This is the general strategy for showing that a combination of objects with various properties will satisfy some conclusion. 