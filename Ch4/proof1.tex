
In this worksheet we have several different exercises to get you used to the CMSC250 styled format of step-by-step proofs.

Here are some guidelines you should take when writing proofs in the 250 style:
\begin{itemize}
    \item When starting the proof, be sure to give all the objects you are working with variable names.
    \item It helps to number your steps, because you might want to refer to a previous step. If this reminds you too much of geometry two column proofs you can label specific steps with a marker like $(*)$ or $(**)$, for example
    \[\text{we have the equation }e^{\pi i} = -1~(*).\]
    \[\text{then by $(*)$ and algebra, $e^{\pi i} + 1 = 0$}.\]
    \item If you are in doubt whether or not to skip a step, I recommend that you do not skip a step.
\end{itemize}

Here are some other guidelines:
\begin{itemize}
    \item Suppose I have a number $n = 2r + 1$. Have I shown that $n$ is odd yet? No! I still need to check that the expression $r$ is an integer. This justification should be one of its own lines in the proof. The lesson here is make sure you have verified \textbf{all parts} of a definition \textit{in writing} before claiming something is of that specific definition type.
    \item What does it mean for an integer $n$ to not be divisible by $3$? It means that there is some integer $q$ where either
    \[n = 3q + 1 \text{ or } n = 3q + 2.\]
    This follows from something called the \textbf{remainder theorem} for integers. What if we replace $3$ with some other number?
    \item In discussion I will show how to structure a proof by cases.
\end{itemize}
 \begin{enumerate}
   \item Suppose $x \in \mathbb{N}^{\leq 4}$ (That is, the set of natural numbers less than or equal to $4$). Then, prove that $n^3 \leq 3^n$.
   \item Prove (by cases) that $100$ is not the perfect cube of a natural number. (Hint: there are two cases to consider. The first problem might help.)
   \item Answer each question, and prove your answer is correct.
\begin{enumerate}
    \item For any $m, n \in \mathbb{Z}$ is $6m + 8n$ even?
    \item For any $m, n \in \mathbb{Z}$ is $10mn + 7$ odd?
    \item For any $c \in \mathbb{Z}$ what can we say about $(c + 1)^2 - (c - 1)^2$?
\end{enumerate}
   \item Suppose $a \in \mathbb{Z}$ is an odd number. Prove that $a^2 = 8m + 1$ for some $m \in \mathbb{Z}$ (that is, $a^2$ is $n$ more than a multiple of $8$). There are many ways to do this problem.
   \item Suppose $m$ is an even integer. Prove that $m(m + 2)$ is a multiple of $8$.
   \item (Challenge Problem) Suppose $p$ is a prime number greater than $3$. (A prime number $p$ is a positive integer which has no positive integer divisors other than $1$ and itself. For example, $2$, $3$, and $101$ are prime, but $15$ is not (since $3$ divides $15$).
Prove that $p^2$ is $1$ more than a multiple of $24$, that is there exists some integer $k$ such that $p^2 = 24k + 1$.
(Hint: at some point you will have to use the result of the previous problem.)
   % add more problem files here
 \end{enumerate}

