Colloquially, when people say that some evidence ``proves'' something, they refer to some object or conversation which demonstrates that a given claim is true. Proof is supposed to be the evidence in favor or demonstrating the claim. In computer science and mathematics, the use of the word ``proof'' is similarly used.
In computer science and mathematics, the general thought process is on various ideas and their consequences. A big focus is on why certain claims are true or false. The reason why there is a big focus on why things are true is that these explanations give some motivation for why other claims are true as well.

In my experience, for very simple claims involving very basic objects, the majority of people can give a reasonable explanation for why such a claim is true or false. For example, take the following claim:

\[\text{The sum of two even numbers is even.}\]

Most people understand an even number to be a number $2n$, where $n$ is an integer. If we have two such even numbers $2n$, and $2m$, then their sum is $2n + 2m = 2(n + m)$. Since $n + m$ is an integer the claim follows. In its most basic form, this is an example of a proof. For our purposes, a proof is nothing more than a logical explanation about why a claim is true. The word ``logical'' is important, because it means that proofs can be either be correct or incorrect. Either all the steps in the explanation logically follow from the previous few, or there is a mistake somewhere, rendering the proof incorrect. Since our definition of ``proof'' is a logical explanation about why a claim is true, then a proof is the best way, in fact the only way, to demonstrate the truth of any given claim.

Since in this book we will introduce several kinds of objects and present some of their properties. Often, showing an object has a property is also a claim, and will require proof. So it makes sense to first introduce the concept of proofs themselves before diving into the math.

In many classes which teach students how to write proofs, a lot of emphasis is given on certain structure and grammatical conventions and tone of language when writing a proof. In reality this sort of emphasis is superfluous. The advantage it gives the instructors is that it makes homework and exams easier to grade. However, that is not to say you can get away with poor writing in proofs, especially if you are presenting proofs for other people to read. 

