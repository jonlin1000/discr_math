In general, there are three kinds of claim ``structures'' which encompass the vast majority of claims that people want to prove. These are the ``object has property'' structure, the ``hypothesis implies conclusion'' structure, and the ``situations are equivalent'' structure. We expand more on these below.

An ``object has property'' structure concerns some specific object or collection of objects, with certain properties known beforehand. The nature of the claim is that the object has some other property, which is usually not totally obvious from the properties that are known beforehand. Here is an example of such a claim:
\begin{proposition}
The set of natural numbers $\mathbb{N}$ are countable.
\end{proposition}
Here the object in question is specific, being the set of natural numbers $\mathbb{N}$ and the property is the property of countability of a set. However, based on the definition of countability, this claim is trivially true based on the definition of countability. Is there anything we might glean from it? Here is another similar claim, but instead with a collection of objects.
\begin{proposition}
Any subset of the natural numbers $\mathbb{N}$ is countable.
\end{proposition}
Here, the property remains the same, but we have expanded to a collection of objects, namely the subset of natural numbers. However, this claim is still trivially true based on the definition of countability. Maybe there is a more general claim which is not immediately obvious? There is, once we replace $\mathbb{N}$ with any countable set.
\begin{proposition}
Any subset of a countable set is countable.
\end{proposition}
We use this set of examples to indicate that a claim involving a very specific object (the natural numbers) can be made into a claim that is much more general (and not immediately obvious). In this way, a lot of mathematics is built from the ``ground up'' from small examples which get more general.

\subsection{Hypothesis implies Conclusion}

The other type of claim usually investigated has a ``hypothesis implies conclusion'' structure. With these claims, the hypothesis usually involves one or several objects with various properties, and the claim is that some given conclusion will follow. That is, these types of claims usually go the following way: ``Suppose A, B, \ldots satisfy the following properties or relations. Then a certain conclusion follows.'' Here is a more specific example:

\begin{proposition}
Suppose $A$, $B$, and $C$ are sets with $A \subset B$ and $B \subset C$. Then $A \subset C$.
\end{proposition}

Here, the objects are $A$, $B$, and $C$ with some properties involving subset relations. The conclusion is another property involving subset relations.

\subsection{Situations are Equivalent}

Similarly to the ``hypothesis implies conclusion'' proof structure, the ``situations are equivalent'' structure involves a set of two or more hypotheses. The claim is that each of these hypotheses implies the other. Usually these claims will have the following form:

\begin{proposition}
The following are equivalent:
\begin{itemize}
	\item Situation A
	\item Situation B
\end{itemize}
\end{proposition}

To prove this proposition is true is the same as showing both the claims ``Situation A implies Situation B'' and ``Situation B implies Situation A''. 

For the situation with $3$ or more hypotheses, one might imagine that one needs to show a lot more than two claims, and the situation may get out of hand quickly. However, this is not the case, since it turns out implication is transitive. That is, if ``Situation A implies Situation B'' and ``Situation B implies Situation C'', then it follows that ``Situation A implies Situation C''. So if we are to show situations A, B, and C are equivalent, one possible way to show this is to show the following claims to be true:

\begin{itemize}
	\item Situation A implies Situation B
	\item Situation B implies Situation C
	\item Situation C implies Situation A.
\end{itemize}

