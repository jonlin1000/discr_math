In this worksheet, we will briefly discuss a property of boolean formulas called \textbf{satisfiability}. This was briefly discussed on the last worksheet, with respect to the boolean formula $p \land (\lnot p)$.

\begin{tcolorbox}
 A boolean formula is \textbf{satisfiable} if there exists an assignment of true or false to each of its variables which makes the formula evaluate to true. Note that we only need \textit{one} such assignment of variables.
 A boolean formula is \textbf{unsatisfiable} if it isn't satisfiable. In particular, all assignments of variables will evaluate to false.
\end{tcolorbox}

For example, the formula $p \land q \land r$ is satisfiable (with satisfying assignment $p \equiv q \equiv r \equiv T$. And this is the ONLY satisfying assignment).
 \begin{enumerate}
   \item Determine whether or not the following Boolean formulae are satisfiable:

\begin{itemize}
    \item $x_1 \land x_2 \land \cdots \land x_n$, and $x_1 \lor x_2 \lor \cdots \lor x_n$, where $n$ is a natural number.
    \item $(x_1 \land (\neg x_2)) \lor (x_2 \land (\neg x_3)) \lor \cdots \lor (x_{n - 1} \land (\neg x_n))$.
    \item $(x_1 \lor \neg x_2) \land (\neg x_1 \lor x_2) \land (x_1 \lor x_2) \land (\neg x_1 \lor \neg x_2)$.
    \item $(x_1 \lor (\neg x_2) \lor x_3) \land (x_1 \lor (\neg x_2) \lor (\neg x_3)) \land ((\neg x_1) \lor x_2 \lor (\neg x_3))$.
\end{itemize}
   \item Explain why a boolean formula with $n$ variables has $2^n$ different assignments.

\begin{enumerate}
    \item Suppose I was to tell you that a certain boolean formula was satisfiable. What would I need to tell/demonstrate to you in order to convince you of this fact? How much time would it take?
    \item Suppose on the other hand I was to tell you that a certain boolean formula was \textit{unsatisfiable}. What would I need to demonstrate to you in order to convince you of this fact? How much time would it take?
\end{enumerate}
   \item Sometimes, despite the computational time it takes to find a satisfying assignment, some unfortunate circumstance may require that you do so. This problem will outline one or two heuristics in order to possibly simplify your job a little further.

Consider the boolean formula

\[(x_1 \land x_5 \land x_2) \lor (x_3 \land (\neg x_3) \land x_5) \lor (x_2 \land (\neg x_7) \land x_6 \land x_1 \land (\neg x_4)).\]

\begin{enumerate}
    \item Notice that this formula is in \textit{disjunctive normal form}. That is, it can be represented as an OR of ANDS. Explain how this might simplify our problem a little bit.
    \item Simplify the formula by considering the variable $x_3$.
    \item Discuss (with your neighbors) the theoretical or practical use of such simplifications.
\end{enumerate}
 \end{enumerate}