Now we prove from the Well Ordering principle the Principle of Mathematical Induction. The idea will be as follows: we will consider the \textit{smallest} element such that a proposition is false, and then we will use the assumptions of the Priniciple of Mathematical Induction to derive contradictions. Hence the Principle of Mathematical Induction is true as stated.

In fact, if we assume the Principle of Mathematical Induction as a given statement, we can use it to show the well ordering principle of the natural numbers. But to give a rigorous proof we need the Principle of Strong Induction, which we will prove from the Principle of Weak Induction later.

\begin{theorem}
Suppose that $P: \mathbb{N} \to \{T, F\}$ is a boolean function with the following properties:
\begin{itemize}
	\item $P(1) \equiv T$
	\item $P(n) \equiv T \implies P(n + 1) \equiv T$ for all $n \in \mathbb{N}$.
\end{itemize}
Then $P(n) \equiv T$ for all $n \in \mathbb{N}$.
\end{theorem}

\begin{proof}
Assume for the sake of contradiction that there is some $s \in \mathbb{N}$ such that $P(s) \equiv F$. Let $S$ be the set of all $s$ where $P(s) \equiv F$. By assumption, $S \neq \varnothing$, and hence by the well ordering principle there is a minimal element $r \in S$. We see that $r \neq 1$ because $P(1) \equiv T$ by assumption. So $r - 1 \in \mathbb{N}$. But since $r \in S$ is minimal, we must have that $P(r - 1) \equiv T$ (or else our assumption about $r$ being the minimal element in $S$ would be wrong). But since $P(r - 1) \equiv T$, this implies that $P(r) \equiv T$ (by modus ponens). This is a contradiction and the conclusion follows.
\end{proof}

To prove the other direction (namely, that the Principle of Mathematical Induction implies the Well Ordering Principle), we need the stronger principle of Strong Induction. So what we will do is use the Principle of Mathematical Induction to prove the Principle of Strong Induction first.

\begin{theorem}
Suppose that the Principle of Mathematical Induction is true. Suppose that $P: \mathbb{N} \to \{T, F\}$ is a boolean function with the following properties:
\begin{itemize}
	\item $P(1) \equiv T$
	\item If $P(k) \equiv T$ for all $1 \leq k \leq n$ for $n \geq 1$, then $P(n + 1) \equiv T$.
\end{itemize}

Then $P(n) \equiv T$ for all $n \in \mathbb{N}$.
\end{theorem}

\begin{proof}
Define an auxiliary function $Q: \mathbb{N} \to \{T, F\}$ by the following:
\[Q(n) \equiv \begin{cases} T & \text{$P(k) \equiv T$ for $1 \leq k \leq n$}\\ F & \text{otherwise.}\end{cases}\]
Then by assumption, $Q(1) \equiv T$ and $Q(k) \equiv T$ implies that $P(n + 1)$ is true, which together with $Q(n)$ implies that $Q(n + 1) \equiv T$. Hence by the Principle of Mathematical Induction we have that $Q(n)$ is true for all $n \in \mathbb{N}$ which implies that $P(n)$ is true for all $n \in \mathbb{N}$, as desired.
\end{proof}

The converse statement is also true, this will be left as an exercise. Now that we have the principle of strong induction, we can use it now to prove the well ordering principle. This will show that the Well Ordering Principle and the Principle of Mathematical Induction are equivalent.

\begin{theorem}
The Strong Induction Principle implies the Well Ordering Principle. Hence the Well Ordering Principle and the Principle of Mathematical Induction are equivalent.
\end{theorem}
\begin{proof}
Suppose that $S \subset \mathbb{N}$ is a non-empty set with no least element. Let $P$ be the following proposition:
\[P(n) = \begin{cases}T & n \not\in S \\ F & n \in S\end{cases}.\]
So the proposition $P(n)$ reflects whether or not $n$ is in the set $S$. Since $1$ is the least element in $\mathbb{N}$, we know that $1 \not\in S$ or else $S$ would have a least element. So $P(1) \equiv T$. Now suppose for any $n \in \mathbb{N}$ that $P(k) \equiv T$ for $1 \leq k \leq n$. Then by the definition of $P$ no natural number less than or equal to $n$ is in $S$. It follows that $n + 1 \not\in S$ or else $n + 1$ would be the least element in $S$. It follows that $P(n + 1) \equiv T$.

By the Principle of Strong Induction it follows that $P(n) \equiv T$ for all $n \in \mathbb{N}$. This implies that no element $n \in \mathbb{N}$ is in $S$. So $S = \varnothing$. The Well Ordering Principle follows, as desired.
\end{proof}

\section{Exercises}
\begin{itemize}
\item Show that the following sets are well-ordered under the ordering described. (It will be helpful to use the fact that $\mathbb{N}$ is well-ordered.
	\begin{itemize}
		\item $\mathbb{N} \cup \{\omega\}$, where $\omega > n$ for any $n \in \mathbb{N}$ and where $\mathbb{N}$ has the usual ordering.
		\item $\mathbb{N} \times \mathbb{N}$ under the dictionary order (which was defined in Example \ref{dict_order}). 
		\item $\mathbb{Z}$ under the following ordering $R$: $aRb$ if $|a| < |b|$, and if $|a| = |b|$ then $a < b$ if $a$ is negative and $b$ is positive.
	\end{itemize}
	
\item Show that the following sets are not well-ordered:
	\begin{itemize}
		\item The set of rationals $\mathbb{Q}$ under the usual ordering.
		\item The set $\{2^n \mid n \in \mathbb{Z}\}$ under the usual ordering on $\mathbb{Q}$.
		\item The cube $[0, 1] \times [0, 1]$ under the dictionary ordering.
	\end{itemize}
	
\item Show that for any well-ordered set $S$ with order relation $<$ and any element $r \in S$, either $r$ is a maximal element ($r \geq s$ for all $s \in S$) or $r$ has an immediate successor $r^{+}$, that is, a smallest element greater than $r$. Is every non-minimal element $r$ an immediate successor of some other element?

\item Suppose that $a$ is any integer and $b$ is any positive integer. State and prove when there exist integers $q$ and $r$ such that $r < |b/2|$ and we have 
\[a = qb + r.\]
When can we ensure that $q$ and $r$ are unique?

\item Prove that the Principle of Strong Induction implies the Principle of Mathematical Induction.

\item Prove the Principle of Strong Induction using the Well Ordering Principle.

\item Suppose that $P(n)$ is a proposition. In each part assume $P(n)$ satisfies the given list of properties and use the Well Ordering Principle to prove that $P(n)$ is true for all $n \geq 0$. After this, generalize.
	\begin{enumerate}
		\item 
		\begin{itemize}
			\item $P(0)$, $P(1)$, and $P(2)$ are true,
			\item $P(n) \equiv T \implies P(n + 3) \equiv T$.
		\end{itemize}
		\item
		\begin{itemize}
			\item $P(0)$ and $P(1)$ are true,
			\item If $P(n)$ is true, then $P(2n)$ is true,
			\item IF $P(n)$ is true, then $P(n - 1)$ is true.
		\end{itemize}
	\end{enumerate}
\end{itemize}