Prove by induction the following equalities for appropriate integers. You, the reader, will have to determine the appropriate base case.
\begin{enumerate}
    \item \[\sum_{k = 1}^nk = \frac{n(n + 1)}{2}.\] (For a solution, peep Jason's lecture!)
    \item \[\sum_{k = 1}^nk(k + 1) = \frac{n(n + 1)(n + 2)}{3}.\]
    \item \[\sum_{k = 1}^nk(k + 1)(k + 2) = \frac{n(n + 1)(n + 2)(n + 3)}{4}.\]
    \item
    
    \begin{enumerate}
        \item \[\sum_{k = 0}^n2^k = 2^{k + 1} - 1.\]
        \item \[\sum_{k = 0}^n3^k = \frac{3^{k + 1} - 1}{2}\]
        \item \[\sum_{k = 0}^n4^k = \frac{4^{k + 1} - 1}{3}.\]
        \item State the pattern you see, and prove this general result using induction.
        \item Using the so called ``high-school'' algebra identity
        \[(x - 1)(x^n + x^{n-1} + \cdots + x + 1) = x^{n + 1} - 1,\] give another proof of your general result. This proof is non-inductive (or is it?).
    \end{enumerate}
\end{enumerate}