Consider the pseudocode for the following recursive function, which takes a list $\verb|ls| = (a_0, \dots, a_n)$ and a function $f$ and returns a list $(f(a_0), f(a_1), \dots, f(a_n))$ for any non-negative integer $n$.

In the code:
\begin{itemize}
    \item \verb|length| is a function returning the length of a list,
    \item \verb|ls[0]| accesses the first element of the list.
    \item \verb|pop(ls)| returns the original list but without the first element. For example, $\verb|pop|(2, 5, 3) = (5, 3)$.
    \item \verb|::| represents a push operator. For example, $2\verb|::|(5, 3)$ will be the list $(2, 5, 3)$.
\end{itemize}

\begin{verbatim}
    function map(ls, f) =
        if (length(ls) = 0) {
            return () /* the empty list. */
        }
        else {
            return f(ls[0])::map(pop(ls), f);
        }
\end{verbatim}

Prove that the code above is \textbf{correct}. That is, given any list $(a_0, \dots, a_n)$ prove that \verb|map| will return $(f(a_0), f(a_1), \dots, f(a_n))$. (Hint: you'll need to use induction. What are you inducting on? Once you have this in mind this problem turns out to be very simple.)