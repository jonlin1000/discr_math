\section{Principle of Strong Induction}

Here is the Principle of Strong Induction:

\begin{theorem}
Suppose $P(n)$ is a proposition, and that the following statements hold:
\begin{itemize}
    \item $P(0)$ is true,
    \item If $P(i)$ is true for all $0 \leq i \leq n$, then we can prove that $P(n + 1)$ is true.
\end{itemize}
Then $P(n)$ holds for all $n \in \mathbb{N}$.
\end{theorem}
\begin{proof}
In fact, this theorem actually directly follows from the Principle of Weak Induction. To see this clearly, define the proposition $Q(n)$ to be equivalent to the claim that $P(i)$ is true for $0 \leq i \leq n$. Then by the hypotheses:
\begin{itemize}
	\item $Q(0)$ is true,
	\item If $Q(n)$ is true, then $P(n + 1)$ is true. This essentially means that if $Q(n)$ is true, then $P(i)$ is true for all $0 \leq i \leq n + 1$, implying that $Q(n + 1)$ is true.
\end{itemize}

So by the Principle of Weak Induction $Q(n)$ is true for all natural numbers $n$, which implies that $P(n)$ is true for all natural numbers $n$ as well.
\end{proof}

%note: define inductive hypothesis in weak_ind.tex

The Principle of Strong Induction is seemingly ``stronger'' than the Principle of Weak Induction. The reason behind this is because of the nature of the theorem's ``induction hypothesis'', which assumes the truth of $P(i)$ for all $i$ up to a value $n$. We may apply this principle to prove certain statements where in order to complete the inductive step assumptions about more than just the previous steps are needed. Here is a particular example of this. 

\section{Exercises}
\begin{enumerate}
    \item Determine whether or not $P(n)$ is true for all $n \geq 0$ given that $P$ satisfies the rules in part (a), part (b), etc. (No proof is needed)

\begin{enumerate}
    \item \begin{itemize}
        \item $P(0)$, $P(1)$, $P(2)$ are true.
        \item If $P(n)$ is true, then $P(n + 3)$ is true.
    \end{itemize}

    \item \begin{itemize}
        \item $P(0)$ and $P(1)$ are true,
        \item If $P(n)$ is true, then $P(2n)$ is true,
        \item If $P(n)$ is true, then $P(n-1)$ is true.
    \end{itemize}
    
    \item Challenge Problem: \begin{itemize}
        \item $P(0)$, $P(1)$ are true.
        \item If $n$ is odd, then $P(n)$ is true if and only if $P(3n + 1)$ is.
        \item If $n$ is even, then $P(n)$ is true if and only if $P(n/2)$ is.
    \end{itemize}
\end{enumerate}
    \item In discussion, I presented the following notion of strong induction (which we called Weak Induction$++$):
\begin{theorem}
Let $P(n)$ be a proposition, and suppose the following statements hold:
\begin{itemize}
    \item $P(0)$ and $P(1)$ is true.
    \item If $P(n-1)$ and $P(n)$ are true for $n \geq 1$, then $P(n + 1)$ is true.
\end{itemize}
The conclusion is that $P(n)$ is true for all $n \in \mathbb{N}$.
\end{theorem}

Generalize this theorem to multiple base cases.
    \item Recall during discussion when we proved that $a_n \leq (7/4)^{n + 1}$, we stated an inductive hypothesis which was similar to that in Theorem 1. Explain why we couldn't state and prove the inductive hypothesis with only one base case $n = 0$, mimicking the format of the above theorem. (Recall that for $n \geq 2$, we had $a_n = a_{n-1} - a_{n-2}$.) 
    \item The following problem outlines a proof of the so called \textbf{arithmetic-geometric mean inequality}: for all non-negative reals $a_1, \dots, a_n$ we have
\[\frac{1}{n}\sum_{i = 1}^n a_i \geq \sqrt[n]{\prod_{i = 1}^na_i}.\]
\begin{enumerate}
    \item When $n = 1$ this inequality is trivially true. For $n = 2$: Note that for all $a > 0$ there is $b > 0$ such that $b^2=a$. Use this fact and the fact that $(x - y)^2 \geq 0$ for all (non-negative) $x$ and $y$ to show directly that the inequality is true when $n = 2$.
    \item Suppose this inequality is true for $n = k$. Using the $n = 2$ case, show that this inequality is true for $n = 2k$. 
    \item Supposing this inequality is true for $n = k$, show that this inequality is true for $n = k-1$.
    \item Conclude that we are done. (This method of proof is called \textbf{Cauchy Induction}.)
\end{enumerate}
    \item Suppose that there are $n$ people that board a flight. They seat themselves one at a time. The first person has forgotten which seat he is assigned, and picks a random seat to sit in. Each subsequent person either sits in their designated seat, or if it is already occupied, they sit in a remaining seat randomly (as in uniformly random).
\begin{enumerate}
	\item When $n = 2$, what is the probability that the last person (person $2$) sits in the seat that he is assigned?
	\item What about when $n = 3$? $n = 4$?
	\item Make a conjecture for general $n$ and prove your conjecture using strong induction.
\end{enumerate}
\end{enumerate}