\section{Summation notation, product notation}

One should read 
\[\sum_{k = m}^na_m\] as the value of \verb|result| after executing this code:
\begin{verbatim}
    result = 0;
    for (k = m; k <= n; k = k + 1) {
        result = result + a_m.
    }
\end{verbatim}

Ideally, most of the time you should see the above sum as just a compactified way to write 
\[a_m + a_{m + 1} + \cdots + a_n\]

Notice that this code does not make sense if $m > n$, because the index $k$ is \textbf{always} incremented. In this case we always take the sum to be $0$ by default. So a sum like
\[\sum_{k = 1}^0k\] is equal to $0$, not $1$.

Similarly for products. one should read \[\prod_{k = m}^na_m\] as the value of \verb|result| after executing this code:

\begin{verbatim}
    result = 1;
    for (k = m; k <= n; k = k + 1) {
        result = result * a_m.
    }
\end{verbatim}

Again this code does not make sense if $m > n$. In this case we always take the product to be $1$ by default.

The following equations are some useful properties related to summations.
\[\sum_{i = m}^n(a_i + b_i) = \sum_{i = m}^na_i + \sum_{i = m}^nb_i\]
The above property is often colloquially referred to as ``splitting a sum''.
\[\sum_{i = m}^nca_i = c\sum_{i = 1}^ma_i\] if $c$ is a constant that does not depend on the index $i$.
\[\sum_{i = m}^na_i = a_m + \sum_{i = m + 1}^na_i\]
\[\sum_{i = m}^na_i = a_n + \sum_{i = n}^{n-1}a_i\]
Note that the above two equations make sense if $n \leq m$. (What happens when $n = m$?)

By yourself come up with some analogous formulas for products.

\section{The Principle of Mathematical Induction}

\begin{definition}
    A proposition $P(n)$ is a logical statement involving some value $n$.
\end{definition}

It is very helpful to consider an example or two. For example, define $P(n)$ to be the logical statement that $n = 1$. In other words, $P(n) \equiv (n = 1)$. Then $P(n) \equiv T$ if and only if $n = 1$, which is true but is silly to say. Another example: define $P(n) \equiv (2 \mid n)$. Then
\[P(n) \equiv \begin{cases}T, & 2 \mid n \\ F, & 2 \nmid n\end{cases}\]

But as of now we aren't really interested in examples like this. We'd like to know if there is a way to show that $P(n) \equiv T$ for $n \geq k$ for some integer $k$, for example.

For example, we'd like to figure out whether or not
\[\sum_{i = 1}^n(2i - 1) = n^2\] for all $n$. It seems true! But how might we prove it? This leads us to a technique called \textit{mathematical induction}, a powerful technique which lets us prove classes of statements indexed by the natural numbers. The technique of mathematical induction essentially relies on the following theorem. We will prove this theorem is true in a future section using a property of the natural numbers called the \textbf{Well-Ordering Principle}.

\begin{theorem}
Let $a$ be any natural number. And let $P(k)$ be a proposition about $k \in \mathbb{N}$. Assume the following claims are true:
\begin{itemize}
    \item $P(a) \equiv T$
    \item Assuming $P(k)$ is true for any natural number $k \geq a$, we can prove that $P(k + 1)$ is true.
\end{itemize}

Then $P(k)$ is true for all $k \geq a$.
\end{theorem}

Essentially, suppose you have an indexed family of propositions $P(n)$ which you want to prove true for $n \geq a$. A proof by mathematical induction is essentially the same as satisfying the hypotheses of the above theorem. Essentially, in a proof by induction you show the following:

\begin{enumerate}
    \item First we show that $P(a) \equiv T$.
    \item Next, assuming that $P(k)$ is true for $k \geq a$, prove that $P(k + 1)$ is true.
\end{enumerate}

Here is an example of this in action. We will prove the first equality stated at the beginning of the chapter.
\begin{proposition}
\[\sum_{i = 1}^n(2i - 1) = n^2\] for all $n$ for all natural numbers $n \geq 1$.
\end{proposition}
\begin{proof}
We will prove this equality using mathematical induction. First, we observe for $n = 1$ that
\[\sum_{i = 1}^1(2i - 1) = 1 = 1^2.\]
Now assume that 
\[\sum_{i = 1}^k(2i - 1) = k^2.\]
Then
\[\sum_{i = 1}^{k + 1}(2i - 1) = \sum_{i = 1}^k(2i - 1) + 2(k + 1) - 1 = k^2 + 2k + 1 = (k + 1)^2,\]
so we are done.
\end{proof}

Let's break down this proof a little bit further. Here the statement $P(n)$ is the statement that the equality 
\[\sum_{i = 1}^n(2i - 1) = n^2\] is true for that specific $n$. In the first part of the proof we show that $P(1)$ is true. In the second part of the proof we show that the truth of $P(k)$ implies the proof of $P(k + 1)$. To do this, we split the sum
\[\sum_{i = 1}^{k + 1}(2i - 1) = \sum_{i = 1}^k(2i - 1) + 2(k + 1) - 1\]
into two parts. By our hypothesis, the left part is exactly $k^2$. The rest of the prooof simplifies the sum to achieve the desired conclusion: that is to show 
\[\sum_{i = 1}^{k + 1}(2i - 1) = (k + 1)^2.\]

\section{Exercises}
\begin{enumerate}
    \item Prove by induction the following equalities for appropriate integers. You, the reader, will have to determine the appropriate base case.
\begin{enumerate}
    \item \[\sum_{k = 1}^nk = \frac{n(n + 1)}{2}.\] (For a solution, peep Jason's lecture!)
    \item \[\sum_{k = 1}^nk(k + 1) = \frac{n(n + 1)(n + 2)}{3}.\]
    \item \[\sum_{k = 1}^nk(k + 1)(k + 2) = \frac{n(n + 1)(n + 2)(n + 3)}{4}.\]
    \item
    
    \begin{enumerate}
        \item \[\sum_{k = 0}^n2^k = 2^{k + 1} - 1.\]
        \item \[\sum_{k = 0}^n3^k = \frac{3^{k + 1} - 1}{2}\]
        \item \[\sum_{k = 0}^n4^k = \frac{4^{k + 1} - 1}{3}.\]
        \item State the pattern you see, and prove this general result using induction.
        \item Using the so called ``high-school'' algebra identity
        \[(x - 1)(x^n + x^{n-1} + \cdots + x + 1) = x^{n + 1} - 1,\] give another proof of your general result. This proof is non-inductive (or is it?).
    \end{enumerate}
\end{enumerate}
    \item Consider the pseudocode for the following recursive function, which takes a list $\verb|ls| = (a_0, \dots, a_n)$ and a function $f$ and returns a list $(f(a_0), f(a_1), \dots, f(a_n))$ for any non-negative integer $n$.

In the code:
\begin{itemize}
    \item \verb|length| is a function returning the length of a list,
    \item \verb|ls[0]| accesses the first element of the list.
    \item \verb|pop(ls)| returns the original list but without the first element. For example, $\verb|pop|(2, 5, 3) = (5, 3)$.
    \item \verb|::| represents a push operator. For example, $2\verb|::|(5, 3)$ will be the list $(2, 5, 3)$.
\end{itemize}

\begin{verbatim}
    function map(ls, f) =
        if (length(ls) = 0) {
            return () /* the empty list. */
        }
        else {
            return f(ls[0])::map(pop(ls), f);
        }
\end{verbatim}

Prove that the code above is \textbf{correct}. That is, given any list $(a_0, \dots, a_n)$ prove that \verb|map| will return $(f(a_0), f(a_1), \dots, f(a_n))$. (Hint: you'll need to use induction. What are you inducting on? Once you have this in mind this problem turns out to be very simple.)
    \item Take a couple of minutes and try to prove by induction that for any integer $n \geq 1$, we have
\[\sum_{k = 1}^n\frac{1}{k^2} < 2.\]
(You won't be able to do it!)

Let's prove something seemingly ``stronger'': Prove by induction that for any integer $n \geq 1$ we have
\[\sum_{k = 1}^n\frac{1}{k^2} < 2 - \frac{1}{n},\] which implies the above result.
    \item Define a function $f(x,y)$, where $x$ and $y$ are allowed to be \textbf{natural numbers.} (which include $0$). Suppose $f$ has the following properties:
\begin{itemize}
    \item $f(n, n) = f(n, 0) = 1$ for all $n \in \mathbb{N}$.
    \item $f(n + 1, k + 1) = f(n, k) + f(n, k + 1)$ for all natural numbers $n$ and $k$.
    \item $f(n, m) = 0$ if $m > n$.
\end{itemize}

Prove by induction (on $n \geq 0$) that
\[\sum_{i = 0}^nf(n, i) = 2^n.\]

(Hint: when proving the inductive step, you will have to split the sum and reindex.)
    \item A \textit{triomino} is an ``L'' shaped tile formed by 3 adjacent squares of a chessboard. We say that an arrangement of triominos is a \textit{tiling} of a chessboard if every square of the chessboard is covered without any triominos overlapping.

Prove by induction that for all integers $n \geq 1$ the $2^n \times 2^n$ sized chessboard missing one square (in ANY location) can be tiled. The fact that $3 \mid (2^n2^n - 1 = 4^n - 1)$ is a consequence of problem 1diii.
    \item (Challenge Problem): Prove Euler's Formula: For any convex polyhedron, the formula
\[v - e + f = 2\] holds, where $v$ is the number of vertices, $e$ is the number of edges, and $f$ is the number of faces. You will want to induct on the number of vertices for $v \geq 4$ (because the notion of ``face'') doesn't make sense for $v < 4$. Your proof will probably not be completely rigorous, and that's okay. (Hints: For the base case, a tetrahedron is the only You cannot solve this problem by just chopping off a vertex, as in general the vertices which are adjacent to the vertex you are attempting to chop off are not generally going to be all in a shared plane. Look up ``edge contraction'' if you are stuck without ideas about the inductive step.)
\end{enumerate}

