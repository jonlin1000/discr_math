Most readers who attended the equivalent of American middle or high school will recall having to solve systems of equations. A general form would be of two by two systems, which look like this:
\begin{align*}
ax + by &= z_1 \\
cx + dy &= z_2.
\end{align*}

In general, we will be dealing with much larger systems of linear equations where the number of variables to be solved for and the number of equations can vary.

\begin{definition}
An $m \times n$ system of linear equations is a system of equations of the form
\begin{align*}
a_{11}x_1 + \cdots + a_{1n}x_n &= y_1 \\
\cdots \\
a_{m1}x_1 + \cdots + a_{mn}x_n &= y_m.
\end{align*}
where the $a_{ij}$ and the $y_k$ are predetermined coefficients in any field (refer to ??). The variables $x_1, \dots, x_n$ are referred to as \textbf{unknowns} or \textbf{unknown variables}. 
\end{definition}

In this chapter, what we will be most focused on are methods that that can determine solutions to these linear equations. That is, we would like to find values $(b_1, \dots, b_n)$ such that all the equations in the definition above are true if we replace $x_i$ with $b_i$ for each $i$. For example, if we have the linear equation 
\begin{align*}
ax + by &= c \\
dx + ey &= f \\
\end{align*}
then $x = g$, $y = h$ is a solution to this system of linear equations (In fact, it's the only solution).

We might first observe the following. Suppose that $(b_1, \dots, b_n)$ is a solution to the $m \times n$ system of linear equations
\begin{align*}
a_{11}x_1 + \cdots + a_{1n}x_n &= y_1 \\
\cdots \\
a_{m1}x_1 + \cdots + a_{mn}x_n &= y_m.
\end{align*}
Then $(b_1, \dots, b_n)$ is also a solution to the equation 
\[\sum_{i = 1}^mc_i(a_{i1}x_1 + \cdots + a_{in}x_n) = \sum_{i = 1}^mc_iy_i\]
where the $c_i$ are arbitrary coefficients. This is an example of a \textbf{linear combination} of the equations in our $m \times n$ system of linear equations. So if we had a second system of linear equations where each equation in the second system was a linear combination of the equations in the first system, then every solution of the first system would be a solution in the second system. This observation is important for proving the following, which is the basis for our methods to solve linear systems of equations:

\begin{theorem}
Suppose $A$ and $B$ are two systems of linear equations which have the property that each equation in one system is a linear combination of equations in the other system. Then $A$ and $B$ have exactly the same solutions.
\end{theorem}

