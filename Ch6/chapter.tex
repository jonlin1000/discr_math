In CMSC250, there are only about two ways in which you might prove a number $\sqrt[n]{q}$ irrational for some $n \in \mathbb{Z}^{>0}$ and some $q \in \mathbb{Z}^{>1}$:

\begin{itemize}
    \item A Euclidean argument. That is, you prove some lemma of the form ``if $q \mid a^n$ then $q \mid a$''.
    \item Use the unique prime factorization theorem and compare the exponents on prime factors modulo $n$.
\end{itemize}

There are two main problems with both approaches:
\begin{itemize}
    \item The Euclidean argument scales poorly to large values of $n$ and large values of $q$, especially when proving a statement like this by contrapositive, where the naive thing to do is to prove separately each remainder case out of the quotient remainder theorem.
    \item Proofs by UPFT are hard to write correctly, are a mess of indices, and are difficult to grade and give feedback. Especially if you have an equation of the form
    \[2p_1^{2e_1}\cdots p_n^{2e_n} = q_1^{2f_1}\cdots q_k^{2f_k},\]
    it is tricky to talk about the exponent of the $2$ in the unique prime factorization of the number on the left hand side.
\end{itemize}

Here in this handout I outline better tools (Bezout's Lemma, and the $p \mid ab$ lemma) which completely resolve these two issues. Proofs using these two methods both scale well to large numbers and are not too hard to write correctly.

\section{Preliminary Definitions and Results}

\begin{definition}
 Given integers $a$ and $b$, we say that $a$ divides $b$ (and we write $a \mid b$) if there exists an integer $k$ such that $a \cdot k = b$.
\end{definition}

\begin{definition}
 Given integers $a$ and $b$ and positive integer $m$ we say that $a$ is congruent to $b$ modulo $m$ (and we write $a \equiv b \pmod{m}$) if $m \mid (a - b)$.
\end{definition}

Here in this document $\mathbb{Z}$ denotes the integers and $\mathbb{N}$ denotes the integers greater than $0$. \textbf{Note that this is different from standard CMSC250 convention!} I'll also abuse the following fact a lot, which is often taken as a given (axiomatic) property of the natural numbers:

\begin{tcolorbox}
 Let $S \subseteq \mathbb{N}$. Then $S$ has a least element. That is, there is an $s \in S$ such that for all $t \in S$, $s \leq t$.
\end{tcolorbox}

This property is called the \textbf{well ordering property} of $\mathbb{N}$. As an aside, note that this property that $T$ is equivalent to the principle of mathematical induction, as proven in the box below.
\begin{tcolorbox}
 To fill in later. It's not too important.
\end{tcolorbox}

\begin{definition}
 Let $a$ and $b$ be \textbf{non-zero} integers. Then the greatest common denominator of $a$ and $b$ (we write $\gcd(a, b)$) is the largest number $g$ which divides $a$ and $b$.
\end{definition}

\begin{example} Here are some simple examples illustrating the gcd.
    \begin{tcolorbox}
      The greatest common denominator of $12$ and $8$ is $4$. \\ For any integer $n$, the greatest common denominator of $n$ and $n + 1$ is $1$. How do we show this? Suppose $a \geq 2$, $a \mid n$ (that is, $ak = n$ for some $k \in \mathbb{Z}$). Then $ak + 1 = n + 1$, and by the quotient remainder theorem it follows that $a$ does not divide $n + 1$. We've deduced that no divisor of $a$ greater than $2$ can also divide $n + 1$. Hence the \textit{greatest} common divisor of $n$ and $n + 1$ is $1$.
    \end{tcolorbox}
\end{example}

\begin{definition}
 A prime number $p$ is a natural number greater than $1$ with the property that the only natural numbers which divide $p$ are $1$ and $p$.
\end{definition}

The following lemma is the key to most irrationality applications. It says that if $p$ is prime and $p \nmid a$, then $p$ and $a$ have no commmon factors other than $1$. Intuitively and formally, this is clear because $p$ only has $2$ factors, $1$ and $p$.
\begin{lemma}
    Suppose $p$ is a prime and $a$ is an integer with $p \nmid a$. Then $\gcd(p, a) = 1$.
\end{lemma}
\begin{proof}
The only divisors of $p$ are $1$ and $p$. The condition $p \nmid a$ shows that $p$ is not a divisor of $a$. Hence $1$ must be the \textit{greatest} common divisor of $p$ and $a$, as desired.
\end{proof}

We also use the following lemma a lot, which is just a way to bound numbers by divisors.
\begin{lemma}
    Suppose $a$ and $b$ are integers, $b$ non-zero, with $a \mid b$. Then $|a| \leq |b|$. In particular, if $a$ and $b$ are positive, then $a \leq b$.
\end{lemma}
\begin{proof}
    $a \mid b$ implies that there is an integer $k$ such that $a\cdot k = b$. Since $b \neq 0$ it follows that $k \neq 0$. So $|k| \geq 1$. It follows that 
    \[|a| = |a| \cdot 1 \leq |a||k| = |ak| = |b|,\] as desired.
\end{proof}

\section{Bezout's Lemma, and the $p \mid ab$ Lemma}

\begin{definition}
 Suppose $m$ and $n$ are integers. An \textbf{integer linear combination} of $m$ and $n$ is any integer of the form 
 \[mx + ny\] where $x, y \in \mathbb{Z}$.
\end{definition}

\begin{example}
    $1$ is an integer linear combination of $13$ and $9$, since
    \[1 = 91 - 90 = 13(7) - 9(10).\]
\end{example}

Given non-zero integers $m$ and $n$, let \[S = \{mx + ny \mid x, y \in \mathbb{Z}, mx + ny > 0\}.\]
In English, $S$ is all the positive integer linear combinations. Since $S \subseteq \mathbb{N}$, it has a least element $\ell$. Bezout's Lemma states that $\ell = \gcd(m, n)$, a surprising result to take in at first.

\begin{lemma}
    Let $m$ and $n$ be non-zero integers, and let $g = \gcd(m, n)$. Then $g$ is the smallest positive integer linear combination of $m$ and $n$, that is, there exist some integers $x_0$, $y_0$ where
    \[g = mx_0 + ny_0,\] and moreover $g$ is the smallest number where such $x_0$ and $y_0$ exist.
\end{lemma}

\begin{proof}
The proof strategy is as follows: define $\ell$ as earlier, that is, as the least positive integer linear combination of $m$ and $n$ (that is, the smallest natural number where $mx_0 + ny_0 = \ell$ for some integer $x_0$ and $y_0$). We prove two intermediate results:
\begin{enumerate}
    \item $\ell \mid n$ and $\ell \mid n$.
    \item $g \mid \ell$
\end{enumerate}

These two things imply $\ell = g$, for the first step tells us $\ell$ is a common divisor of $m$ and $n$, and the second step tells us that the \textit{greatest} common divisor divides $\ell$, hence is less than or equal to $\ell$. But $g$ is the greatest common divisor, so $g = \ell$.

Let's prove these two steps. Suppose (by contradiction), $\ell \nmid m$. Then by the quotient remainder theorem there exist integers $q$ and $r$ where $0 < r < \ell$ such that $m = q\ell + r$. Then we have
\[r = m - \ell q = m - (mx_0 + ny_0)\ell = m(1 - x_0\ell) + n(b_0\ell).\]
Both expressions in the parentheses in the right hand side are integers by closure under addition and multiplication. So we have expressed $r > 0$, strictly less than $\ell$, as an integer linear combination of $m$ and $n$. But $\ell$ was defined as the \textbf{least} integer linear combination! This is a contradiction, so $\ell \mid m$. Similarly, $\ell \mid n$.

Let $g = \gcd(m, n)$. There exist $k_1, k_2 \in \mathbb{Z}$ such that $gk_1 = m$ and $gk_2 = n$. Then using the $x_0$ and $y_0$ as above we get
\[\ell = mx_0 + ny_0 = g(k_1x_0 + k_2y_0),\]
which shows that $g \mid \ell$, as desired.
\end{proof}

What immediately follows is a very nice lemma about a prime dividing a product.

\begin{lemma}
    Let $p$ be a prime and $a$ and $b$ be integers. If $p \mid ab$ then $p \mid a$ or $p \mid b$.
\end{lemma}
\begin{proof}
    To prove this statement we need to show that the statement
    \[(p \mid a) \lor (p \mid b).\]
    
    is always true. To prove this or statement, there are two cases. Either $(p \mid a)$, which means the or statement is true. So there is nothing to do here. So suppose $p \nmid a$. We need to show that $(p \mid b)$ to show that the or statement is true.
    
    If $p \nmid a$ then $\gcd(a, p) = 1$. So there exist integers $x, y$ such that $ax + py = 1$. Multiplying this equation by $b$, we get $abx + pby = b$. But since $p \mid ab$ there exists an integer $k$ such that $pk = ab$. So substituting this into the above equation we get
    \[pkx + pby = p(kx + by) = b,\] and since $kx + by$ is an integer it follows that $p \mid b$, as desired.
\end{proof}

\section{Immediate applications to irrationality proofs}

Using the $p \mid ab$ lemma we can immediately deduce the Euclidean lemma for any prime and any exponent, as highlighted below.

\begin{lemma}
Suppose $p$ is a prime and $a$ is an integer with $p \mid a^2$. Then $p \mid a$.
\end{lemma}
I'll give two different proofs of this fact, one using Bezout's Lemma and one using the $p \mid ab$ lemma.
\begin{proof}
    By contrapositive. Suppose $p \nmid a$. Then $\gcd(p, a) = 1$, so there exist integers $x$ and $y$ such that $px + ay = 1$. Squaring this equation we get 
    \[p^2x^2 + 2pxay + a^2y^2 = p(px^2 + 2pxy) + a^2(y^2) = 1.\]
    This shows that $1$ is an integer linear combination of $p$ and $a^2$. Since it is the smallest possible integer linear combination we can conclude that $\gcd(a^2, p) = 1$. That is,
    $p \nmid a^2$, as desired.
\end{proof}
Isn't that neat? No nightmare of indices. No dealing with $p-1$ cases (especially as $p$ gets very large). Just square the equation $ax + py = 1$ to get what you want. The second proof is even easier.

\begin{proof}
    Suppose $p \mid a^2$. Then $p \mid a \cdot a$. So $p \mid a$ or $p \mid a$, which means that $p \mid a$, as desired.
\end{proof}

In fact, this proof is so nice that it leads the way to an easy proof by induction.

\begin{lemma} \label{gen_euclid}
Suppose that $p$ is a prime and $a$ is an integer with $p \mid a^n$. Then $p \mid a$.
\end{lemma}

\begin{proof}
We'll prove this by induction on $n$. For the case $n = 1$, clearly $p \mid a$ implies $p \mid a$. Now assume that for $n \geq 1$ that $p \mid a^n$ implies $p \mid a$. Suppose $p \mid a^{n + 1}$. Then $p \mid a$ or $p \mid a^n$. Clearly if $p \mid a$ we would be done. Even if we suppose not, then $p \mid a^n$ and our inductive step gives us $p \mid a$ anyway. So the inductive step is complete and we are done.
\end{proof}

The next lemma shows that mod a prime, any non-zero integer has an inverse modulo $p$.

\begin{lemma} \label{inverse}
Suppose $p$ is any prime, and $k$ is not a multiple of $p$. Then there exists an integer $\ell$ such that 
\[k \ell \equiv 1 \pmod{p}.\]
\end{lemma}

\begin{proof}
Since $k$ is not a multiple of $p$, it follows that $\gcd(k, p) = 1$. So there exist integers $x$ and $y$ such that $kx - py = 1$, or $kx - 1 = py$. This implies that $p \mid (kx - 1)$, or $kx \equiv 1 \pmod{p}$, as desired.
\end{proof}

\begin{example}
We'll find all the solutions $n$ to the equation $5n \equiv 3 \pmod{7}$.
\begin{tcolorbox}
 By Lemma \ref{inverse} there exists some $k$ such that $5k \equiv 1 \pmod{7}$. After some searching we see that $k = 3$ is an integer such that $5k \equiv 1 \pmod{7}$. Hence
 \[5kn \equiv 3k \pmod{7} \implies n \equiv 3(3) \equiv 2 \pmod{7}.\]
 
 So all such $n$ must be of the form $7k + 2$ for some integer $k$. Conversely, if $n = 7k + 2$, then 
 \[5n = 35k + 10 = 7(5k + 1) + 3,\] or $5n \equiv 3 \pmod{7}$. This shows that all solutions $n$ to the equation are the integers $7k + 2$ where $k$ is \textbf{any} integer.
\end{tcolorbox}
\end{example}

\section{Exercises}
 \begin{enumerate}
   \item Suppose $m$ and $n$ are integers. Then consider the set $S$ of all $k$ such that $k$ is an integer linear combination of $x$ and $y$. Let $g = \gcd(m, n)$.
\begin{enumerate}
    \item Suppose $g \mid k$. Prove that there are $x, y \in \mathbb{Z}$ such that $mx + ny = k$.
    \item Suppose $g \nmid k$. Prove (by contradiction) that there are no $x, y \in \mathbb{Z}$ such that $mx + ny = k$. Hence the only integer linear combinations of $m$ and $n$ are multiples of $g$.
\end{enumerate}
   \item In discussion, I presented the following notion of strong induction (which we called Weak Induction$++$):
\begin{theorem}
Let $P(n)$ be a proposition, and suppose the following statements hold:
\begin{itemize}
    \item $P(0)$ and $P(1)$ is true.
    \item If $P(n-1)$ and $P(n)$ are true for $n \geq 1$, then $P(n + 1)$ is true.
\end{itemize}
The conclusion is that $P(n)$ is true for all $n \in \mathbb{N}$.
\end{theorem}

Generalize this theorem to multiple base cases.
   \item Recall during discussion when we proved that $a_n \leq (7/4)^{n + 1}$, we stated an inductive hypothesis which was similar to that in Theorem 1. Explain why we couldn't state and prove the inductive hypothesis with only one base case $n = 0$, mimicking the format of the above theorem. (Recall that for $n \geq 2$, we had $a_n = a_{n-1} - a_{n-2}$.) 
   \item Let $S \in \mathbb{Z}$ be a non-empty (this is important!) set such that 
\begin{itemize}
    \item $(\forall x, y \in S)[x + y \in S]$ (Closure under addition)
    \item $(\forall x \in S)[-x \in S]$. (Existence of additive inverse)
\end{itemize}

Show that there exists $g \in S$ such that $g \mid k$ for all $k \in S$. Describe $S$ in terms of $g$. (Hint: consider the smallest element $\ell$ of $S \cap \mathbb{N}$). Why did I specify that $S$ be non-empty?
   \item Prove that for all positive real numbers $r$ and $s$ that $\sqrt{r + s} \neq \sqrt{r} + \sqrt{s}$. So the non-identity called the ``freshman dream'' is not true.
   \item In this problem we outline a proof of the \textbf{Fundamental Theorem of Arithmetic}: the statement that every positive integer greater than $1$ factors uniquely into primes.
\begin{enumerate}
	\item Use induction to show that for a prime $p$ and integers $a_1, \dots, a_n$ that if $p \mid a_1\cdot \dots\cdot a_n$, then $p \mid a_i$ for some $1 \leq i \leq n$. 
	\item This fact implies unique factorization in the following sense: Suppose that $n$ is an integer with two different factorizations. Take a prime $p$ in one factorization. Then prove that $p$ must occur in the other factorization. (For an elegant proof, use the well ordering principle to derive a contradiction.)
\end{enumerate}
   \item In this problem we will outline how one can write a computer program that can parse and evaluate logical expressions such as $\shortsim(T \lor F) \land (F \lor T)$. For the sake of simplicity, we may assume that any logical expression only consists of the standard and ($\land$), or ($\lor$), and not ($\shortsim$) operators.

\begin{enumerate}
	\item Scan the input string and form a ``token list'' of all the individual relevant characters. For example, an expression of the form $\shortsim(T \land F) \lor F$ can be decomposed into the form  
\begin{verbatim}
	[~, (, T, ^, F, ), v, F]
\end{verbatim}
	\item Assuming that your expression is well founded, one can make what is known as a \textit{context free grammar}, which is a set of rules that recursively define your expression.
	\item Write a set of computer functions that call each other and parse your token list, based on your recursive expression definition. In the language you are programming in you will need to make structures that are of the form $f(a, b)$, in functional notation. For example, in python, one could potentially use heterogeneous arrays, for example,
\begin{verbatim}
[``and'', [``or'', true, false], false]
\end{verbatim}
to represent these expressions. The resulting expression that comes out is known as a ``syntax tree'', and closely models the functional expressions described in the text.
	\item Write a function that will recursively evaluate a syntax tree that has been written, as described in the chapter.
\end{enumerate}
   % add more problem files here
 \end{enumerate}
