Mods are a tricky concept to get right because they rely on internalizing concepts that you've learned earlier very well. Once you've internalized those concepts you'll see that modular arithmetic is really just a recasting of divisibility. In this handout, I'd like to highlight the correspondence between divisibility, mods, and the quotient remainder theorem.

\begin{tcolorbox}
\textbf{Quotient Remainder Theorem}: Suppose that $n$ is an integer and $m$ is a positive integer. Then there exist \textit{unique} integers $q$ and $r$ with $0 \leq r < m$ where
\[n = qm + r.\]
\end{tcolorbox}

\begin{tcolorbox}
Let $a$ and $b$ be any integers. We say that $a$ divides $b$ (and we write $a \mid b$) if there exists an integer $k$ such that 
\[a \cdot k = b.\]
\end{tcolorbox}

We are going to define mods in terms of divisibility. To motivate the definition a little bit, when we take mods we want to have that two numbers are equivalent modulo $n$ if they have the same remainder (as in the quotient remainder theorem) when they are divided out by $n$. So for example, $15 \equiv 3 \pmod{4}$ because $15 = 4(3) + 3$. Here is the definition:
\begin{tcolorbox}
Let $a$ and $b$ be integers, and let $n$ be a positive integer. We say that $a$ is congruent to $b$ modulo $n$ (and we write $a \equiv b \pmod{n}$) if $n \mid (a - b)$.
\end{tcolorbox}

Let's pick apart this definition a little bit. In the above example, we have $15 \equiv 3 \pmod{4}$. Is that true? It is, since we see that $15 - 3 = 12 = 4(3)$, so $4 \mid (15 - 3)$, as we wanted. (If you are confused, I am taking $a = 15$, $b = 3$ in this example).

To get you used to the proof approaches using modular arithmetic I will do a simple proof involving mods below.

\begin{tcolorbox}
$a \equiv b \pmod{n}$ if and only if $a - b \equiv 0 \pmod{n}$.
 \begin{tcolorbox}
  Proof:
  \begin{enumerate}
      \item Suppose that $a \equiv b \pmod{n}$.
      \item By definition, $n \mid (a - b)$.
      \item By algebra, $a - b = (a - b) - 0$.
      \item By substituting $(3)$ into $(2)$, 
      \[n \mid (a-b) - 0\].
      \item By definition $a - b \equiv 0 \pmod{n}$, as desired.
      
      \item The converse is left as an exercise to the reader.
  \end{enumerate}
 \end{tcolorbox}
\end{tcolorbox}

There are three important properties of mods which you will use in modular arithmetic problems. I provide proofs for two of these below.

\begin{tcolorbox}
    If $a \equiv b \pmod{n}$ and $c \equiv d \pmod{n}$, then $a + c \equiv b + d \pmod{n}$.
    \begin{tcolorbox}
        Proof:
        \begin{enumerate}
            \item Suppose the above.
            \item By definition, $n \mid (a-b)$ and $n \mid (c-d)$.
            \item By definition of divisibility, there exist integers $k_1$ and $k_2$ such that $a - b = nk_1$ and $c - d = nk_2$.
            \item By (3) and algebra, we add the two equations together to get 
            \[(a - b) + (c - d) = n(k_1 + k_2).\]
            \item Rearranging the LHS we get
            \[(a + c) - (b + d) = n(k_1 + k_2).\]
            \item $k_1, k_2 \in \mathbb{Z}$ by (3). So by closure $k_1 + k_2 \in \mathbb{Z}$.
            \item By (6), (5), definition of mods, it follows that $a + c \equiv b + d \pmod{n}$, as desired.
        \end{enumerate}
    \end{tcolorbox}
\end{tcolorbox}

\begin{tcolorbox}
 If $a \equiv b \pmod{n}$ and $c$ is any integer then $ca \equiv cb \pmod{n}$.
 \begin{tcolorbox}
  Proof: left as exercise for the reader.
 \end{tcolorbox}
\end{tcolorbox}

\begin{tcolorbox}
 If $a \equiv b \pmod{n}$ and $c \equiv d \pmod{n}$ then $ac \equiv bd \pmod{n}$.
 \begin{tcolorbox}
  \begin{enumerate}
      \item Given the above:
      \item By the first thing proved in this handout, $a - b \equiv 0 \pmod{n}$ and $c - d \equiv 0 \pmod{n}$.
      \item By one of the above facts, we have $c(a - b) \equiv 0 \pmod{n}$ and $b(c-d) \equiv 0 \pmod{n}$ (because $c \times 0 = 0$ and similarly with $b$).
      \item By addition of mods, we get
      \[c(a-b) + b(c-d) \equiv 0 \pmod{n}\]
      \item By algebra on the left hand side we get
      \[ac - bd \equiv 0 \pmod{n}\]
      \item This gives us $ac \equiv bd \pmod{n}$, as desired.
      
  \end{enumerate}
 \end{tcolorbox}
\end{tcolorbox}

Finally, we'll recast the remainder theorem in terms of mods. From the equation $n = qm + r$, we get $n - r = qm$, so we have $m \mid n - r \iff n \equiv r \pmod{m}$. With this in mind, here is the reformulation of the quotient remainder theorem:

\begin{tcolorbox}
 \textbf{QRT in terms of modular arithmetic}: Suppose that $n$ is any integer and $m$ is a positive integer. Then there exists a unique integer $r$ where $0 \leq r \leq m$ and 
 \[n \equiv r \pmod{m}.\]
\end{tcolorbox}

Below are the problems. Everyone should attempt problem 2 (and 3a). Everything else is more challenging and should be attempted at your own discretion.

 \begin{enumerate}
   \item Write out a proof of the equivalence of definitions of the very last box. That is, suppose that the quotient remainder theorem is true. Then, using the definition of mods as given, show that the assertion made in the last box holds as a result. Conversely, assume that the statement in the last box holds, and use it to give a proof of the quotient remainder theorem.
   \item Suppose that $a$ is any integer. Then show that either $a^2 \equiv 0 \pmod{4}$ or $a^2 \equiv 1 \pmod{4}$. (In light of the quotient remainder theorem it follows that both cannot happen at the same time.)
   \item
   \begin{enumerate}
       \item Suppose that $S =  \{1, 2, 3, 4\}$. For any $k \in S$, prove that there exists $\ell \in S$ such that $k \cdot \ell \equiv 1 \pmod{5}$. (Hint: this is a finite domain problem.)
       \item (\textbf{Challenge Problem*}) Suppose that $S = \{1, 2, \dots, p-1\}$ for $p$ a prime number. Then show that for all $k \in S$ there exists $\ell \in S$ such that $k \cdot \ell \equiv 1 \pmod{p}$.
       \item (\textbf{Challenge Problem***}) Prove Wilson's Theorem: For any prime $p$,
\[(p - 1)! \equiv -1 \pmod{p}.\] Here $n! = n \times (n-1) \times \cdots \times 2 \times 1$ for any positive integer $n$.
   \end{enumerate}
   \item (\textbf{Challenge Problem}) Suppose $ab \equiv ac \pmod{d}$. Then prove that $b \equiv c \pmod{\frac{d}{\gcd(a, d)}}$. Here $\gcd(a, d)$ is the \textit{greatest common divisor} of $a$ and $d$, that is, the greatest integer $k$ such that $k \mid a$ and $k \mid d$.
   \item Prove that there is no positive integer $n$ such that $4n^2 - 4 \equiv 0 \pmod{19}$.
 \end{enumerate}