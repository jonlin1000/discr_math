In this section we prove the observation that the Infinite Ramsey Theorem implies the Finite Ramsey Theorem. The idea we will use is a common argument called a ``subsequence argument'': passing (many times) to subsequences we will find a sequence of complete (but finite graphs) each of which avoid a coloring we are avoiding and are related to each other. In this way we find a coloring of the complete graph on $\mathbb{N}$ which avoids this coloring as well.

\begin{theorem}
Assume that the following theorem is true: If $K_\mathbb{N}$ is edge-colored with two colors, then there exists an infinite subset $S \subset \mathbb{N}$ such that the subgraph $K_S$ is monochromatic. Then the following theorem is true. For all $n \in \mathbb{N}$ there exists $N(n) \in \mathbb{N}$ such that if $K_N$ is edge-colored red or blue then $K_N$ contains a monochromatic $K_n$.
\end{theorem}

\begin{proof}
We will assume the contrary. Then we will be done if we construct a coloring of $K_N$ which does not contain a monochromatic infinite $K_S$.

Suppose now this is true. Then there exists $n \in \mathbb{N}$ such that for all $N \in \mathbb{N}$ there is a edge-coloring $J_N$ of $K_N$ which avoids a monochromatic $K_n$. To make choices consistent suppose we label $K_N$'s vertices $\{1, 2, \dots, N\}$. Now we find a ``nice'' subsequence of these colors as follows:

\begin{enumerate}
	\item First enumerate the set of edge pairs $\{a, b\} \subset \mathbb{N}$ as $e_i$, which is possible because this set is countable. 
	\item For $e_1$, for each graph where this edge exists, we can find a subsequence $J_{i_1}, J_{i_2}, \dots$ where in each coloring where applicable the edge $e_i$ is colored the same. (We can include the colorings where $e_i$ does not exist, since there are only finitely many of these).
	\item Do the same inductively for $e_2, e_3, \dots$.
\end{enumerate}
In the end we have a subsequence of colorings
\[J_{j_1}, J_{j_2}, \dots\]
with the following properties:
\begin{itemize}
	\item The $J_{j_k}$ avoid any monochromatic $K_n$,
	\item For any edge $e_i = \{a, b\}$ every $J_{j_k}$ colors $e_i$ the same color.
\end{itemize}

We can use this to define a coloring of $K_\mathbb{N}$ as follows. For each edge $e_i$ we color this edge by the unique color that is assigned to it by the $J_{j_k}$. It is clear that this coloring avoids a monochromatic $K_n$. Indeed, we observe that if such a coloring on the edges between the vertices $m_1, \dots, m_n$ was monochromatic then we can find some $j_k > \max(m_1, \dots, m_n)$ and we observe that this cannot be monochromatic. So we are done now since a monochromatic $K_S$ where $S$ is infinite necessarily contains a monochromatic $K_n$.
\end{proof}